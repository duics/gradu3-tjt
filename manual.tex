\documentclass[a4paper,12pt]{article}
\usepackage[T1]{fontenc}
\usepackage[latin1]{inputenc}
\usepackage{url}

\newenvironment{cmds}
  {\begin{description}\renewcommand{\makelabel}[1]{\texttt{\textbackslash##1}\hfill}}
  {\end{description}}

\newenvironment{opts}
  {\begin{description}\renewcommand{\makelabel}[1]{\texttt{##1}\hfill}}
  {\end{description}}

\newenvironment{envs}
  {\begin{description}\renewcommand{\makelabel}[1]{\texttt{##1}\hfill}}
  {\end{description}}

\newcommand{\cmd}[1]{\texttt{\textbackslash#1}}
\newcommand{\opt}[1]{\texttt{#1}}
\newcommand{\env}[1]{\texttt{#1}}
\newcommand{\pack}[1]{\textit{#1}}
\newcommand{\argu}[1]{\textit{#1}}
\newcommand{\cls}[1]{\textsc{#1}}
\newcommand{\prog}[1]{\textbf{\textsf{#1}}}
\newcommand{\file}[1]{\texttt{#1}}

\author{Matthieu Weber\\ \texttt{mweber@mit.jyu.fi} \and Antti-Juhani Kaijanaho\\ \texttt{antkaij@mit.jyu.fi}}
\title{\cls{Gradu3} --- A Class for Writing Master's Thesis at the Faculty of
Information Technology of the University of Jyv�skyl�}

\begin{document}
\maketitle

\begin{abstract}
  This class is designed to facilitate the writing of a Master's
  Thesis (and other thesis) work at the Department of Information
  Technology of the University of Jyv�skyl�.  It may also be used
  elsewhere if it conforms to their requirements.
\end{abstract}

\section{Read this first}

The class is provided currently on a ''beta test'' basis.  We urge you
to consider it your honor-bound duty to \emph{let us know of any
deficiencies} that you may find when using this class.  First,
however, please verify that any strangeness you may witness is not
mandated by the university publishing unit specifications!

While this class has already been used to typeset several subsequently
approved theses successfully, \emph{it is always your responsibility,
as the author,} to ensure that the document meets the requirements
set by the university and the faculty in question. This class aids in
that, but it does not, and cannot, implement the all of the
typesetting requirements.

Please, subscribe to the mailing list Tutkielma-TeX
(\url{http://lists.jyu.fi/mailman/listinfo/tutkielma-tex}), and ask
there any questions you might have.  The subscribers there may be able
to help you faster than we can.

We may make changes to the class from time to time.  New releases will
be announced on that mailing list.

\section{Invocation}

Invoke the class with the usual \cmd{documentclass\{gradu3\}}.  The
options discussed in the following subsections should be given to the
\cmd{documentclass} command as optional arguments
(e.g. \cmd{documentclass[utf8,english]\{gradu3\}}).

\subsection{Language Options}

Languages options set hyphenation, titles translations and date formats to the
selected language. The available languages are:

\begin{opts}
\item[finnish] switches to Finnish language (default)
\item[english] switches to English language
\end{opts}

You can switch at any time from one of those languages to the other one with
\cmd{selectlanguage\{}\textit{language}\texttt{\}}.

\subsection{Input Character Set}

Input character set options allow to specify with what character set the source file
is written. Available character sets are:

\begin{opts}
\item[ascii] ASCII encoding (ISO 646)
\item[utf8] UTF-8 Unicode encoding (ISO 10646)
\item[latin1] ISO-8859-1 encoding (Western Europe languages) (default)
\item[latin2] ISO-8859-2 encoding (Central Europe languages)
\item[latin3] ISO-8859-3 encoding (Esperanto, Maltese)
\item[latin5] ISO-8859-5 encoding (Cyrillic)
\item[latin9] ISO-8859-15 encoding (Western European languages with the Euro symbol)
\item[applemac] Macintosh encoding
\item[ansinew] Windows 3.11 ANSI (extended ISO-8859-1) encoding
\item[cp1252] synonym for \opt{ansinew}
\item[cp1250] MS Windows 1250 (central and eastern Europe languages) code page
\item[decmulti] DEC Multinational Character Set encoding
\item[next] Next encoding
\item[cp437] IBM 437 code page
\item[cp437de] IBM 437 code page (German version)
\item[cp850] IBM 850 code page
\item[cp852] IBM 852 code page
\item[cp865] IBM 865 code page
\end{opts}

Nowadays, \opt{utf8} is almost always a safe choice, and it is the
default in \cls{gradu3}.

Note that these options are given to the \cmd{documentclass} command.
You should not invoke \cmd{usepackage\{inputenc\}} yourself!



\section{Other options}

\begin{opts}
\item[kandi] This is a Bachelor's thesis.
\end{opts}

\section{Header Commands}

The following commands can be used before the \cmd{begin\{document\}}. Some
of those are optional (their default values are described along with the
command) and the others are mandatory. If one of the mandatory commands is not
used, a reminder will be printed inside the document.

\begin{cmds}
\item[title] document's title (mandatory)
\item[author] documents's author (mandatory).  This command can be
  repeated to specify more than one author.
\item[ohjaaja] thesis supervisor (mandatory).  This command can be
  repeated to specify more than one supervisor.
\item[date] document's publishing date (defaults to the current date)
  \textbf{deprecated, use \textbackslash setdate instead}
\item[setdate] takes three arguments: day, month, year (all numeric);
  sets document's publishing date (defaults to the current date)
\item[yliopisto] the university to which this thesis will be
  submitted, in Finnish (defaults to ``Jyv�skyl�n yliopisto'')
\item[university] the university to which this thesis will be
  submitted, in English (defaults to ``University of Jyv�skyl�'')
\item[laitos] the department  to which this thesis will be submitted (defaults to ``Tietotekniikan laitos'' in Finnish and ``Department of Mathematical Information Technology'' in English)
\item[aine] the formal subject (major or minor) whose thesis this is
  (defaults to ``Tietotekniikan'' in Finnish and ``in Information
  Technology'' in English
\item[linja] study line (optional)
\item[tyyppi] type of work in Finnish (defaults to ``pro gradu -tutkielma'')
\item[type] type of work in English (defaults to ``A Master's Thesis'')
\item[keywords] document's keywords in English (mandatory)
\item[avainsanat] document's keywords in Finnish (mandatory)
\item[contactinformation] author's contact information (mandatory). Can be
specified multiple times for multiple authors.
\item[yhteystiedot] author's contact information (synonym of
\cmd{contactinformation})
\item[tiivistelma] abstract in Finnish (mandatory)
\item[abstract] abstract in English (mandatory)
\item[translatedtitle] title in the other language (mandatory)
\end{cmds}

\section{Sectioning Commands}

The available sectioning commands are:
\begin{cmds}
\item[preface] Preface
\item[termlist] List of terms
\item[mainmatter] Marks the beginning of the main part of the document.
Should appear before the first \cmd{chapter}.
Inserts the table of contents, the list of tables and the list of figures.
Switches on chapter numbering; switches the page numbering from
lowercase roman to arabic, starting from 1 again.
\item[chapter] Beginning of a chapter
\item[section] Beginning of a section
\item[subsection] Beginning of a subsection
\end{cmds}

Note that \cmd{subsubsection}, \cmd{paragraph} and \cmd{subparagraph} are
forbidden.

In addition to the previous commands, there are three useful environments:

\begin{envs}
\item[thebibliography] allows to specify a list of references. Should be put
before the appendices.
\item[appendix] inserts an ``Appendices'' line into the table of contents and
the chapters inside the environments will be numbered with
letters instead of numbers, beginning from ``A''.
\item[chapterquote] allows you to add a quote to the beginning of a
  chapter.  It takes one mandatory argument: the attribution line for
  the quote.  You should start the text of the chapter immediately
  after this environment; do not leave empty lines there.
\end{envs}

\appendix
\section{Usage changes from \cls{gradu2} to \cls{gradu3}}

The old \cls{gradu2} class had a lot of options to cater for differing
tastes in the face of very underspecified departmental requirements.
Since \cls{gradu3} is based on new stricter requirements, there is no
longer need for so many options\footnote{Remember, the more options a
  class has, the easier it is to break.}.

Therefore, the following options supported by \cls{gradu2} are no
longer recognized: \opt{altlinespread}, \opt{altsubsec},
\opt{copyright}, \opt{logo}, \opt{lof}, \opt{lot}, \opt{newtitle},
\opt{nocopyright}, \opt{nonumbib}, \opt{nottoc}, \opt{numbib},
\opt{oldtitle}, \opt{palatino}, \opt{shortthesis}, \opt{surnamefirst},
and \opt{toc}.  Similarly, the commands \cmd{acmccs},
\cmd{copyrightowner}, \cmd{copyrightyear}, \cmd{fulltitle},
\cmd{license}, \cmd{paikka}, \cmd{setauthor}, and \cmd{ysa} have been
removed.  Further, the conditional \cmd{ifgradu@pdf} has been removed
(users should let \pack{hyperref} figure out the correct driver by
itself).

In \cls{gradu3}, use the standard \LaTeX{} command \cmd{author} to
specify the thesis author.  Unlike in standard \LaTeX, however,
multiple authors should be specified by giving each their own
\cmd{author} command; \cmd{and} may not be used.  This also makes the
(now removed) option \opt{surnamefirst} obsolete, as the class no
longer cares what part of the name contains the surname.

The following commands are new in \cls{gradu3}: \cmd{ohjaaja}.

\end{document}
