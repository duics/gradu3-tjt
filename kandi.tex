\documentclass[finnish,utf8,palatino,kandi]{gradu3}

\makeatletter
\ifgradu@pdf
  \usepackage[pdftex,bookmarksopen,bookmarksnumbered]{hyperref}
	%\hypersetup{colorlinks,citecolor=blue}
\else
	\RequirePackage[hypertex]{hyperref}
\fi
\makeatother

\title{P ja NP ovat samat}
\author{Antti-Juhani Kaijanaho}
\yhteystiedot{antti-juhani@kaijanaho.fi}
\translatedtitle{P is NP}
\abstract{This is my english abstract}
\tiivistelma{Tämä on suomenkielinen tiivistelmä}
\keywords{P, NP, complexity theory}
\avainsanat{P, NP, kompleksisuusteoria}

\begin{document}
\preface
Esipuhe, jos sellaisen haluaa

\termlist
P: polynomisessa ajassa ratkeavien ongelmien joukko \\
NP: niiden ongelmien joukko, joiden ratkaisu on tarkistettavissa polynomisessa ajassa

\mainmatter

\section{Johdanto}

Johdannon alussa on hyvä esittää tutkielman keskeinen väite
(engl.~\emph{thesis statement}) yhdessä virkkeessä.

Muutoin johdannossa voi kertoa tutkielman aiheen taustoja -- miksi
aihe on kiinnostava (toisaalta yleisesti ja toisaalta tekijän itsensä
näkökulmasta.  Johdannon ei yleensä ole syytä olla sivua pidempi.

Huomaathan, että tässä esimerkissä esitetyt ohjeet saattavat poiketa
ohjaajasi kannasta.  Noudata ensisijaisesti ohjaajasi ohjeita.

Johdannon lopuksi on tapana kertoa, miten tutkielma jäsentyy lukuihin.
Pelkkää suorasanaista sisällysluetteloa kannattaa kuitenkin välttää.

\section{Kirjallisuuskatsaus}

Tutkielmaan sisältyy yleensä jonkinlainen kirjallisuuskatsaus.  Ideana
on tarkastella tutkimuskirjallisuuteen nojaten, mitä tutkielman
aihepiiristä on aiemmin kirjoitettu ja mitä tästä voidaan päätellä.
Esimerkiksi
Taivalsaari~\cite{taivalsaari93:_critic_view_inher_reusab_objec_progr}
esitti uuden olio-ohjelmoinnin mallin, Kevon, jossa tavanomaisesta
poiketen ei ole lainkaan luokkia (Kevo ei kuitenkaan ollut ensimmäinen
tällainen).

Tavoitteena tällaisessa kirjallisuuskatsauksessa tulisi olla osoittaa
jokin laadukas tutkimus, jota kandityössä pyritään matkimaan.
Mahdollista on myös yrittää osoittaa, ettei tällaista tutkimusta ole
aiemmin tehty (mutta tällöinkin kandityön työläyden pitämiseksi
aisoissa olisi kuitenkin hyvä löytää läheltä liippaavia töitä, joista
voi ottaa kevyesti mallia).

\section{Menetelmät}

Seuraavaksi tutkielmassa tarkastellaan kirjallisuuteen tukeutuen
menetelmiä, joita tutkielmassa käytetään.  Tällä tarkoitetaan paitsi
tutkimusmenetelmiä myös tutkimuksen kohteena olevia teknologioita.

\section{Tulokset}

Seuraavaksi tutkielmassa esitellään saadut tutkimustulokset, lyhyesti
ja ytimekkäästi (mutta selkeästi).

\section{Tarkastelu}

Sitten tutkielman tekijä peilaa tuloksia kirjallisuutta ja esittää
tulkintoja.

\section{Johtopäätökset}

Lopuksi lyhyessä johtopäätösosassa kerrotaan tutkimuksen pihvi.


\bibliographystyle{finplain} % tai joku muu, sovi ohjaajasi kanssa
\bibliography{esim}

\appendix

\section{Ensimmäinen liite}

\section{Toinen liite}


\end{document}
