\documentclass[a4paper,12pt]{article}
\usepackage[T1]{fontenc}
\usepackage[latin1]{inputenc}

\newenvironment{cmds}
  {\begin{description}\renewcommand{\makelabel}[1]{\texttt{\textbackslash##1}\hfill}}
  {\end{description}}

\newenvironment{opts}
  {\begin{description}\renewcommand{\makelabel}[1]{\texttt{##1}\hfill}}
  {\end{description}}

\newenvironment{envs}
  {\begin{description}\renewcommand{\makelabel}[1]{\texttt{##1}\hfill}}
  {\end{description}}

\newcommand{\cmd}[1]{\texttt{\textbackslash#1}}
\newcommand{\opt}[1]{\texttt{#1}}
\newcommand{\env}[1]{\texttt{#1}}

\author{Matthieu Weber\\ \texttt{mweber@mit.jyu.fi} \and Antti-Juhani Kaijanaho\\ \texttt{antkaij@mit.jyu.fi}}
\title{Gradu --- A Class for Writing Master's Thesis at the Faculty of
Information Technology of the University of Jyv�skyl�}

\begin{document}
\maketitle

\begin{abstract}
  This class is designed to facilitate the writing of a Master's
  Thesis (and other thesis) work at the Department of Information
  Technology of the University of Jyv�skyl�.  It may also be used
  elsewhere if it conforms to their requirements.
\end{abstract}

\section{Invocation}

Invoke the class with the usual \cmd{documentclass\{gradu2\}}.

\subsection{Language Options}

Languages options set hyphenation, titles translations and date formats to the
selected language. The available languages are:

\begin{opts}
\item[finnish] switches to Finnish language (default)
\item[english] switches to English language
\end{opts}

You can switch at any time from one of those languages to the other one with
\cmd{selectlanguage\{}\textit{language}\texttt{\}}.

\subsection{Input Character Set}

Input character set options allow to specify with what character set the source file
is written. Available character sets are:

\begin{opts}
\item[ascii] ASCII encoding (ISO 646)
\item[latin1] ISO-8859-1 encoding (Western Europe languages) (default)
\item[latin2] ISO-8859-2 encoding (Central Europe languages)
\item[latin3] ISO-8859-3 encoding (Esperanto, Maltese)
\item[latin5] ISO-8859-5 encoding (Cyrillic)
\item[latin9] ISO-8859-15 encoding (Western European languages with the Euro symbol)
\item[applemac] Macintosh encoding
\item[ansinew] Windows 3.11 ANSI (extended ISO-8859-1) encoding
\item[cp1252] synonym for \opt{ansinew}
\item[cp1250] MS Windows 1250 (central and eastern Europe languages) code page
\item[decmulti] DEC Multinational Character Set encoding
\item[next] Next encoding
\item[cp437] IBM 437 code page
\item[cp437de] IBM 437 code page (German version)
\item[cp850] IBM 850 code page
\item[cp852] IBM 852 code page
\item[cp865] IBM 865 code page
\end{opts}

Usual encodings, by platform, are:

\begin{description}
\item[Linux] \texttt{latin1} (or nowadays more often than not, \texttt{latin9})
\item[MS Windows] \texttt{cp1252}
\item[Macintosh] \texttt{applemac}
\end{description}

\section{Copyrights}

The class outputs copyright assertion by default.  This can be inhibited.

\begin{opts}
\item[copyright] Output copyright assertion and license information on
  the second title page (default).
\item[nocopyright] Do not output copyright assertion and license
  information.
\end{opts}

\section{Other options}

\begin{opts}
\item[numbib] Number bibliography (default).
\item[nonumbib] Do not number bibliography.
\item[newtitle] A different title page format (attempts to conform to CS dept requirements)
\item[shortthesis] This is a short thesis; no chapters will be used.
  (For Bachelor's theses etc.)
\item[altsubsec] Alternate layout for subsections (the first line of the
paragraph is on the same line as the title of the subsection)
\end{opts}

\section{Header Commands}

The following commands can be used before the \cmd{begin\{document\}}. Some
of those are optional (their default values are described along with the
command) and the others are mandatory. If one of the mandatory commands is not
used, a reminder will be printed inside the document.

\begin{cmds}
\item[title] document's title (mandatory)
\item[author] document's author (two arguments: first names and surname) (mandatory)
\item[date] document's publishing date (defaults to the current date)
  \textbf{deprecated, use \textbackslash setdate instead}
\item[setdate] takes three arguments: day, month, year (all numeric);
  sets document's publishing date (defaults to the current date)
\item[yliopisto] the university to which this thesis will be
  submitted, in Finnish (defaults to ``Jyv�skyl�n yliopisto'')
\item[yliopisto] the university to which this thesis will be
  submitted, in English (defaults to ``University of Jyv�skyl�'')
\item[laitos] the department  to which this thesis will be submitted (defaults to ``Tietotekniikan laitos'' in Finnish and ``Department of Mathematical Information Technology'' in English)
\item[aine] the formal subject (major or minor) whose thesis this is
  (defaults to ``Tietotekniikan'' in Finnish and ``in Information
  Technology'' in English
\item[linja] study line (optional)
\item[tyyppi] type of work in Finnish (defaults to ``pro gradu -tutkielma'')
\item[type] type of work in English (defaults to ``A Master's Thesis'')
\item[keywords] document's keywords in English (mandatory)
\item[avainsanat] document's keywords in Finnish (mandatory)
\item[contactinformation] author's contact information (mandatory)
\item[yhteystiedot] author's contact information (synonym of
\cmd{contactinformation})
\item[tiivistelma] abstract in Finnish (mandatory)
\item[abstract] abstract in English (mandatory)
\item[translatedtitle] title in the other language (mandatory)
\item[copyrightowner] owner of the copyright (defaults to the author's name)
\item[license] license under which the document is published (defaults to ``All
rights reserved.'')
\item[copyrightyear] copyright year (defaults to the current year)
\end{cmds}

\section{Sectioning Commands}

The available sectioning commands are:
\begin{cmds}
\item[preface] Preface
\item[termlist] List of terms
\item[mainmatter] Marks the beginning of the main part of the document.
Should appear before the first \cmd{chapter}.
Inserts the table of contents, the list of tables and the list of figures.
Switches on chapter numbering; switches the page numbering from
lowercase roman to arabic, starting from 1 again.
\item[chapter] Beginning of a chapter
\item[section] Beginning of a section
\item[subsection] Beginning of a subsection
\end{cmds}

Note that \cmd{subsubsection}, \cmd{paragraph} and \cmd{subparagraph} are
forbidden.

In addition to the previous commands, there are three useful environments:

\begin{envs}
\item[thebibliography] allows to specify a list of references. Should be put
before the appendices.
\item[appendix] inserts an ``Appendices'' line into the table of contents and
the chapters inside the environments will be numbered with
letters instead of numbers, beginning from ``A''.
\item[chapterquote] allows you to add a quote to the beginning of a
  chapter.  It takes one mandatory argument: the attribution line for
  the quote.  You should start the text of the chapter immediately
  after this environment; do not leave empty lines there.
\end{envs}


\end{document}
